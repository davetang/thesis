Recente ontwikkelingen op het gebied van DNA sequentie analyse hebben geleid tot zogenaamde ‘high-throughput sequencing’. Dit heeft DNA sequentie analyse mogelijk gemaakt die zowel tijds- als kostenbesparend werkt.

Met het gereedkomen van de sequentie van het humane genoom en de sequentie van de genomen van andere species, is de nadruk bij genoom sequentie analyse komen te liggen op de identificatie van kleine functionele elementen in het DNA.  Het veld van transcriptoom analyse heeft zich voornamelijk gericht op de correcte annotatie en data analyse van de getranscribeerde producten. De introductie van high-throughput sequencing in transcriptoom analyse heeft een unbiased en accurate screening van het transcriptoom mogelijk gemaakt, en heeft daarmee inzicht geboden in de grote complexiteit van bijvoorbeeld het humane genoom. Echter alle methoden zijn nog relatief nieuw en niet gematureerd, eigenlijk allemaal ontwikkeld in de laatste 5-6 jaar. Daarom vinden er continu verbeteringen aan deze methoden plaats. In mijn dissertatie besteed ik daarom veel aandacht aan het gebruik van bio-informatica methoden om het transcriptoom beter te kunnen analyseren.

In hoofdstuk 2 richt ik mij daarom op de mogelijke bias en technische artefacten  die bij de transcriptoom analyse voor kunnen komen, in het bijzonder die bij de ‘nano-CAGE’ analyse kunnen optreden. In het bijzonder vond ik artefacten die werden geintroduceerd door het gebruik van moleculaire barcodes en ik heb verschillende methoden ontwikkeld om hier om op een juiste wijze mee om te gaan.

In hoofdstuk 3 heb ik mij gericht op het analyseren van het transcriptoom van cellen met geinduceerde DNA schade gebruik makend van sequentie analyse van ‘small RNAs’. Hierbij ontdekten we een nieuwe vorm van RNA molekulen, die worden gevormd dicht op de plaatsen van DNA schade. Deze RNAs vormen de respons op de DNA schade.

In hoofdstuk 4 heb ik me gericht op het bestuderen van transcripten die ontstaan uit repetitieve elementen (RE) in het DNA. Ik heb daarbij het expressie profiel van deze transcripten voor een groot aantal cellijnen, weefsels, en primaire cellen in kaart gebracht. Ik heb daaruit geconcludeerd dat RE’s belangrijk zijn voor de expressie van de ‘long non-coding RNAs’ en ‘enhancer RNAs’, waarbij hun expressie tissue specifiek is.

In conclusie, high-throughput sequentie analyse heeft het veld van transcriptoom analyse wezenlijk veranderd door veel meer complexiteit van transcriptie aan te tonen dan daarvoor voor mogelijk werd gehouden. De lagen van complexiteit die we daarbij hebben ontdekt omvatten meerdere klassen van RNA’s, ieder met een eigen regulatoire rol, random transcriptionele gebeurtenissen ook wel ‘random noise’ genoemd, en inzicht in de complexiteit van de technische artefacten. Het scheiden van technische variatie van het echte signaal vergt begrip van de verschillende technologieen, diepgaande data analyse, en de ontwikkeling van adequate bio-informatische methoden.
