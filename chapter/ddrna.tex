It has been previously reported that small RNAs play a role in establishing the DNA damage response\cite{pmid19444217}. We leveraged the discovery potential of small RNA sequencing to gain some insight into the potential interplay between small RNAs and DNA damage. Intriguingly, we discovered that small RNAs formed near the site of DNA damage. We considered alternative hypotheses that could explain the presence of these small RNAs such as RNA degradation or technical artefacts. We\footnote{All the bioinformatic analysis was performed by Dave Tang.} characterised certain features of the DNA damage small RNAs (DDRNAs) to the population of sequenced small RNAs and observed that the bulk of the DDRNAs had a characteristic size and nucleotide profile, suggesting that they are not random degradation products. Furthermore, we performed experiments to suggest possible biogenesis pathways and the knock-down of Dicer and Drosha, which are enzymes involved with small RNA processing, affected the profile of DDRNAs and impaired the DNA damage response. This work has important implications to the observation that the genome is pervasively transcribed, as it offers a hypothesis that pervasive transcription is necessary for maintaining DNA integrity.
