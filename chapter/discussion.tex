The work carried out as part of this thesis had the underlying goal of interpreting the transcriptional output from various biological systems. In the various studies, different transcriptome profiling methods employing the use of high-throughput sequencers were used to quantify the expression of transcripts. These methods provided unbiased profiling, as \textit{a priori} knowledge is not required,, and provides a better quantification of transcript expression. However the interpretation of transcriptional output is a complex problem due to the various components that affect the output; these fall broadly into two categories: technical and biological variation. In order to assess the biological variation, technical variations such as experimental noise and biases must be accounted for or removed. While technical variation is predictable, biological variation is much harder to assess\cite{pmid22769017}.

The identification of artefacts and biases is important prior to the analysis of data and while tools\cite{pmid19737799, fastqc} are available for performing quality control, they are unable to deal with idiosyncratic biases from specific protocols. In chapter ~\ref{template_switching} we demonstrated a particular type of bias that was introduced with the use of molecular barcodes. DNA barcodes are easily introduced by modifying the sequences of primers or linkers and are usually used to allow different RNA sources to be pooled at an early stage of sample preparation. This has two main benefits in that the starting material is increased and different samples can be sequenced on a single flowcell. However, we observed that the efficiency of primer annealing was dependent on the barcode sequence that was chosen\cite{Tang01022013}, creating a barcode bias. Furthermore, we noticed a particular type of artefact that was produced with the template-switching step in the experimental protocol\cite{Tang01022013}. Specifically, these template-switching artefact, which are a result of strand invasion, interrupted the process of reverse transcription when the cDNA has increased sequence complementary to the template-switching primer. We developed a computational method to remove these artefacts and improved the experimental protocol.

There have been several surprises in the genomics field since the completion of the draft human genome sequence \cite{lander2001initial,venter2001sequence}. The first was the number of genes in the genome, which many genome biologists had overestimated\cite{genesweep}. However, it should be pointed out that this was not a surprise to some; based on theories of mutational load, Susumu Ohno had correctly predicted that the number of functional loci to be under 30,000\cite{pmid5065367}. The next surprise was the observation that most of the genome was transcribed\cite{pmid11988577, pmid17571346}, which became known as pervasive transcription. There were claims that the genome is not as pervasively transcribed as suggested by tiling arrays\cite{pmid20502517}, however this has been robustly reproduced using various transcriptome profiling methods\cite{pmid22955620}. The question is whether products of pervasive transcription are functional or not, especially given the low fidelity of RNA polymerase II\cite{pmid17277804}. Ultimately, the test for functionality will rely on detailed experiments that investigate individual targets.

A reoccurring theme in large scale transcriptome projects using unbiased high-throughput technologies is the discovery of unexpected genome features, such as new species of RNAs. In chapter ~\ref{ddrna}, we discovered a new class of RNAs involved with the DNA damage response (DDR), which we christened as DDRNAs\cite{francia2012site}. These RNAs were independently discovered by another group, who named them diRNAs\cite{pmid22445173}. The DDR is a coordinated set of events that repairs lesions in DNA; the DNA break is detected by specialised factors that initiates a signalling cascade from the site of DNA damage\cite{pmid19847258}. We showed that the DDR is sensitive to RNase A treatment and that Dicer and Drosha products can restore the DDR in RNase A treated cells. We performed small RNA sequencing on cells with an exogenous locus that can be induced to be cut, mimicking DNA damage, to discover the DDRNAs. The DDRNAs were produced from the site of DNA damage but were captured in a limited quantity, which suggests either that they are lowly expressed, or quickly degraded after DNA repair. In order to demonstrate that DDRNAs were either not products of degradation or spurious transcription, we showed that they had size and nucleotide profiles that were different from other classes of small RNAs. Furthermore, when these DDRNAs were either chemically synthesised or generated \textit{in vitro}, the DDR in RNase A treated cells could be restored.

Major portions of eukaryotic genomes are occupied by repetitive DNA sequences, which mainly consists of transposable elements (TEs). The presence of TEs has impacted the human genome in various ways such as generating insertion mutations and genomic instability\cite{pmid19763152,pmid11700292} to shaping gene regulatory systems\cite{pmid18368054}; in fact it was proposed that TEs still pervade in genomes because they are a rich source of genetic material\cite{pmid18368054}. In chapter ~\ref{repeat}, we quantified the transcriptional output arising from repetitive elements to investigate their contribution to the transcriptome. As previously reported, transcription initiating within TEs is more tissue-specific with lower expression levels than protein-coding transcripts\cite{pmid19377475}. However, we observed that specific families of repeats were enriched in particular cell types.

The classical view of transcription initiation was that transcription began at a single position at TA-rich regions known as the TATA-box. One of the major findings made using the CAGE technology was that not all transcription initiation events occurred at a single position\cite{pmid16645617}. While the classical TATA-box promoters mostly initiated from a single position (which were termed sharp promoters), promoters that were CG-rich (in particular CpG islands) showed initiation events across a stretch of sequence (these were termed broad promoters). Initially these CAGE tags were thought to be noise, however, these TSSs were highly consistent in orthologous mouse and human promoters, and sharp and broad promoters were consistently detected in various libraries\cite{pmid16645617}. In addition, this initial survey of the transcriptional landscape in mammalian genomes identified many novel mRNAs and non-coding RNAs that had not been previously characterised\cite{pmid16141072}. CAGE has also been applied to study the dynamics of TSS usage throughout a time course of growth arrest and differentiation\cite{pmid19377474}.

\cite{pmid25218058}.

Various criteria have been used as a criteria for functionality, in contrast to noise, in transcriptional products; these can be broadly separated into \textit{in silico} and experimental validations. \textit{In silico} methods are based on analysing properties associated with an expressed transcript. The expression strength is a commonly used criteria, as consistently captured signal is less likely to occur just by chance, i.e. the smaller the margin of error with an increase in signal. Context-specific transcription, such as observed expression at a specific time-point, developmental stage, or tissue type, can also be used to support function. The argument is such that a random transcript would not become active under a specific condition. However, the criteria of context-specific expression patterns can be countered by the fact that different cell types and during different developmental stages, express a specific repertoire of transcription factors and are under certain circumstances that would lead to the context-specific transcription. Sequence conservation is usually used to support functionality, as this is evidence for purifying or negative selection. For example, protein-coding genes are typically very well conserved among different organisms. Lastly, transcriptional data is typically shown in conjunction with other genome-wide assays.

Kelly\cite{pmid23181609}.

The historical use of poly-A tail selection for enriching mRNAs and removing rRNA contamination in RNA preparations has led to the under-sampling of non-polyadenylated transcripts. The use of random primers in the CAGE protocols, circumvents this bias and allows the capture of poly-A+ and poly-A- RNAs. To investigate the non-coding portion of the genome the CAGE technology was cruicial.

\begin{itemize}
   \item Complex picture behind transcript expression, the products of the genome
   \item The number of transcripts can be quantified to get some idea of the level of transcript expression
   \item Apart from understanding which transcripts are expressed, expression profiling allows the molecular classification of diseases such as cancer, which can lead to the development of biomarker tests   \item Building gene networks from transcriptional profiles
   \item Sequencing depth and sampling of RNA molecules; absolute transcript quantification will help (such as using unique molecule identifiers and non-PCR based methods)
\end{itemize}

Transcripts of Unknown Function (TUFs). The role of non-coding RNAs. The role of repetitive elements in the genome.

Small non-coding RNAs have been implicated in many biological processes, such as messenger RNA regulation or transposon silencing. We identified a role of small non-coding RNAs in the DNA damage repair mechanism. The inactivation of dicer and drosha, which are required for the biogenesis of small RNAs, leads to a loss of DNA damage repair. We sequenced the small RNAs from cells that had induced DNA damage and found small RNAs arising from the vicinity of the DNA double-strand break. Importantly, synthetic small RNAs mimicking these small RNAs could drive the DNA damage response \cite{francia2012site}.

Epigenetic roles such as guiding chromatin-modifying enzymes to their sites of action or acting as scaffolds.

\begin{itemize}
   \item Unbiased profiling of total transcripts
   \item Comparison with different environmental conditions
   \item Gene ontology enrichment
   \item Inaccuracies of gene models
   \item template-free activity of reverse transcriptase, leads to an additional G nucleotide to the 5' end
\end{itemize}

For example, in an early benchmark of RNA-Seq using the Solexa/Illumina sequencing technology, replicate sequencing runs on the same samples demonstrated that technical variations from sequencing were small\cite{pmid18550803}.

The first indication of the importance of non-polyadenylated transcripts\cite{pmid4213457}.

Starting RNA amount required by different protocols and it may not be possible obtain sufficient amounts of starting material. Thus PCR-based techniques are popular and are able to amplify RNA isolated from single cells. This leads to less heterogeneity compared to techniques that pool cells, which may have very different expression profiles. It is important to obtain a true profile of transcription in cells. PCR amplification is necessary for fluorescent-based imaging. PCR bias from GC and AT frequencies.

Cytoplasmic versus nuclear enrichment. The complexity of nuclear transcripts determined by denaturation-renaturation experiments\cite{pmid5969070}

If our assumption is correct, there's one other feature of the spurious transcription that must be observed: the transcription will be cell specific or developmentally regulated. This is because different transcription factors are present in different cell types and at different stages of development. It's also because the accessibility of different parts of the genome vary from cell type to cell type and at different kinds of development. This is the transition from "open" chromatin to a "closed" version resembling heterochromatin.

One criterion used to suggest function is context specificity of an RNA transcript. This includes tissue specificity, developmental stage specificity, or stimulus-specific activation, such as in response to stress.

The underlying assumption is that the proximity of sequences to structurally significant sites implies a regulatory role

Definition of a promoter in light of CAGE.

CAGE versus RNA-Seq versus microarrays\cite{pmid24676093} Data interpretation and the limitation of CAGE, which is that we do not know the length of the transcripts.

Many ncRNA have low evolutionary conservation and are lowly transcribed, and have been dismissed as technical artifacts and/or background transcription that have no biological importance. Alternatively spliced lncRNA\cite{Johnsson2013}.

Are the majority of detected low-level transcription due to technical artifacts and/or background biological noise? Discussion about sequencing depth and saturation. Targeted RNA sequencing reveals the deep complexity of the human transcriptome\cite{pmid22081020}.

Viable mouse with deletion of ultra-conserved regions\cite{pmid17803355} or Neat1\cite{pmid21444682}.

Find evidence for function through comparative analyses and statistical methods (test for certain biases).


TEs are a major genetic component of all organisms and a major contributor to the mutation process. It is currently estimated that 70-80% of spontaneous mutations are the result of TE-mediated insertions, deletions, or chromosomal rearrangements. Thus, it seems at least plausible that TEs may play a significant role in the adaptation and evolution of natural populations and species.

The ubiquity of TEs suggests that they are an old component of genomes which have been vertically transmitted through generations over evolutionary time.

The relationship between TEs and their host genomes:

Are TEs like parasites or symbionts coevolving with their host genome? Do TEs constitute a genetic load for the host genome? Are TEs able to sequester host genes for their own evolution? Are host genomes able to sequester TE sequences for host functions? How can TEs be transferred from one species to another and what is the frequency of this phenomenon?

LTR retrotransposons and the evolution of eukaryotic enhancers - http://www.ncbi.nlm.nih.gov/pubmed/9440254

Since LTR retrotransposons and retroviruses are especially prone to regional duplication and recombination events, these viral-like systems may be especially conducive to the evolution of closely spaced combinatorial regulatory motifs. Using the Drosophila copia LTR retrotransposon as a model, we show that a regulatory region contained within the element’s untranslated leader region (ULR) consists of multiple copies of an 8 bp motif (TTGTGAAA) with similarity to the core sequence of the SV40 enhancer.

Retrotransposons are the most abundant and widely distributed class of eukaryotic transposable elements. Retrotransposon insertions adjacent to chromosomal genes frequently result in altered regulatory phenotypes. For example, retrotransposon insertions into a gene’s 5’ flanking region may affect transcriptional initiation in a temporal-specific or tissue-specific manner. Such regulatory changes can be due to the read-through of transcripts initiated in the retrotransposon promoter or to the presence of positive or negative regulatory sequences within the element. Another way in which retrotransposons may influence gene expression is through insertion induced changes in chromatin structure which may, for example, insulate a gene’s promoter from enhancer sequences located distal to the site of insertion.

Although the hypothesis that transposable elements may have a dramatic effect on regulatory evolution was first proposed by McClintock over 40 years ago, it is only recently that experimental evidence has begun to accumulate which directly supports the hypothesis. Instances of retrotransposons contributing to the evolution of chromosome gene regulation have recently been described in vertebrates, Drosophila and plants. Because it now seems likely that the regulatory evolution of at least some chromosomal genes has been influenced by retrotransposon insertions, the question arises as to what factors may be influencing the evolution of retrotransposon regulatory sequences in the first place.

LTR retrotransposons are closely related to infectious retroviruses and encode genes homologous to the retroviral gag and pol genes flanked by long terminal repeats (LTRs). Because of the similarity between retroviruses and LTR retrotransposons with regard to their structure and regulatory controls, we will group them in this paper under the collective heading “LTR retroelements”. The cis-regulatory sequences contained within LTR retroelement LTRs and adjacent untranslated leader regions (ULRs) interact with host encoded regulatory proteins to control LTR retroelement expression.

Enhancers are cis-acting sequences that increase the utilisation of promoters usually in a tissue and/or developmental specific manner. Enhancers characteristically consist of a series of short repeated sequence motifs that are often associated with regulatory protein binding domains. For example, the well-studied enhancer of the simian virus-40 (SV-40) early genes consists of a 6 bp DNA sequence motif (CCGCCC) which is repeated six times. The rabbit beta-globin enhancer consists of two adjacent 14 bp sequence motifs. The repeated motifs within enhancers are usually binding sites for regulatory proteins and the strength of an enhancer (i.e., the relative effect the enhancer has on promoter initiation) is often positively correlated with the number of repeating motifs it contains.

The characteristic pattern of repeating motifs present within LTR retroelement enhancers is a characteristic by-product of reverse transcriptase mediated replication

Examination of the LTR retrotransposon enhancers shown in Figure 2 reveals three characteristic patterns of repeats. The simplest pattern is a series of short randem sequence motifs as present within the HIV-1, VL30, TNT-1, and gypsy enhancers. An intermediate pattern of complexity consisting of a series of short tandem motifs repeated as a unit two or more times is exemplified within the gypsy enhancer. An example of a more complex pattern in which two or more adjacent heterologous motifs are repeated as a group two or more times is present within the VL30 enhancer. Each one of these patterns can be generated during LTR retrotransposon replication.

Reverse transcriptase (RT) mediated LTR retroelement replication is a highly error prone process with no proofreading ability. One common error in the reverse transcriptase process is the generation of short regional duplications. In addition, frequent recombination events are also known to occur between the two genomic RNA strands packages within LTR retroelement capsids. Regional mispairing between these two RNA templates and/or template switching errors during reverse transcription are capable of generating the more complex patterns found in many retroviral and LTR retrotransposon enhancers. LTR retroelements may also be subject to unequal ectopic recombination events between repeating elements within a genome. Such unequal DNA level exchanges could also contribute to the generation of the complex motif patterns seen in many retroviral and LTR retrotransposon enhancers.

The Drosophila copia LTR retroelement is a model system for the study of LTR retroelement enhancer evolution

Because LTR retroelements may be continually generating variation within their non-encoding enhancer regions, continuous opportunities may exist for natural selection to favour the evolution of adaptive enhancer configurations. This hypothesis rests upon the assumption that at least some of the structural variability being generated within LTR retroelement enhancer regions provides a reproductive advantage upon which natural selection can act. We have begun to address this and related issues concerning the evolution of LTR retroelement enhancers within the context of the Drosophila copia LTR retroelement.

Copia RNA levels are variable among Drosophila species

Transcription is a major rate-limiting step in the retrotransposition process. Thus, naturally occurring genetic variation that influences the transcription of retrotransposons may be of evolutionary significance. We previously reported that steady state levels of copia RNA in Drosophila adults varies significantly among D. melanogaster populations. In that same study it was reported that non copia transcripts are detectable in D. simulans or D. mauritiana adults. We recently reported that transcript levels in larvae follow these same trends. Variation in copia transcript levels between D. melanogaster populations can vary ~ 30-fold, whereas no transcripts are detectable in other melanogaster group species nor in D. willistoni.

Naturally occurring variation in the number of repeating motifs within the copia LTR-ULR is of functional significance

The 9 bp repeating motif within the copia ULR is a binding site for at least two host-encoded proteins. A computer search revealed a significant similarity between the 9 bp motif that is repeated seven times within the full-length copia ULR and the consensus C/EBP (CCAAT/enhancer binding protein) binding site. C/EBP is a mammalian transcriptional activator. It has recently been reported that C/EBP is a trans-regulator of the human LTR retroelement, HIV-1.

The full-length copia ULR is capable of enhancing expression of a heterologous chromosomal gene

The results summarised up to now indicate that naturally occurring copia variants lacking a 28 bp region within their ULR have significantly reduced expression relative to variants containing the 28 bp region. To determine whether the copia ULR with or without the 28 bp gap has properties characteristic of a eukaryotic enhancer, we conducted transient expression assays to test if a copia full-length ULR and a ULR lacking the 28 bp region could influence the ability of a heterologous hsp 70 minimal promoter to drive expression of a bacterial CAT reporter gene in D. melanogaster. The results summarised in Figure 6 indicate that when the full-length ULR is placed upstream of the hsp 70 minimal promoter, it stimulates expression four-fold relative to the control. The ULR lacking the 28 bp region, on the other hand, had no significant enhancer effect on the hsp 70 minimal reporter relative to the control.

We propose that LTR retrotransposons may constitute a mechanism by which eukaryotic enhancers evolve and are subsequently distributed throughout eukaryotic genomes over evolutionary time. This model is based on three hypotheses: 1) that reverse transcriptase mediated LTR retroelement replication is prone to the generation of short regional duplications that are characteristic of eukaryotic enhancers; 2) that at least some of this size variation within LTR/ULRs can lead to the emergence of new LTR retroelement enhancers and/or to an increase in the strength of existing enhancers thereby providing an opportunity for selection; 3) that enhancers that evolve by this mechanism within LTR retroelements also have the potential to exert cis-regulatory effects on the expression of chromosomal genes when distributed throughout the genome by retrotransposition.

Recent findings indicate that eukaryotic enhancers act by facilitating the formation of stable domains within which promoter activity is permitted. Although the effect of enhancers on the expression of extrachromosomal gene constructs, as monitored in transient assays, is considered to be the same as for constructs stably integrated within chromosomes, the later situation is clearly more complex. Stably integrated genetic constructs have been shown to be relatively more efficient at creating a transcriptionally active state within regions of inactive chromatin when they contain an enhancer. By implication, we envision that LTR retrotransposons with the strongest enhancers may have the highest probability of being expressed in any chromosomal context.

