\setlength{\parskip}{\baselineskip}%
\setlength{\parindent}{0pt}%

Small non-coding RNAs have been implicated in many biological processes, such as messenger RNA regulation or transposon silencing. We identified a role of small non-coding RNAs in the DNA damage repair mechanism. The inactivation of dicer and drosha, which are required for the biogenesis of small RNAs, leads to a loss of DNA damage repair. We sequenced the small RNAs from cells that had induced DNA damage and found small RNAs arising from the vicinity of the DNA double-strand break. Importantly, synthetic small RNAs mimicking these small RNAs could drive the DNA damage response \cite{francia2012site}.

Parkinson’s disease (PD) is a slowly progressive disease in which dopamine neurons in the substantia nigra degenerate undetected for years before clinical symptoms develop. The lack of clinical symptoms highlights the necessity of a laboratory test, such as an assay for biomarkers, which can correlate subjects with PD risk. We have profiled the RNAs in the whole blood sample of PD patients and age-matched controls using high-throughput deepCAGE sequencing. By comparing the RNA profiles between PD patients and controls, we aim to discover novel biomarkers that are present in whole blood, which may be further developed into a non-invasive clinical test for PD.

\subsection{Experimental differences}

Starting RNA amount required by different protocols and it may not be possible obtain sufficient amounts of starting material. Thus PCR-based techniques are popular and are able to amplify RNA isolated from single cells. This leads to less heterogeneity compared to techniques that pool cells, which may have very different expression profiles. It is important to obtain a true profile of transcription in cells.

poly-A versus random primers
cytoplasmic versus nuclear enrichment
CAGE versus RNA-seq

\subsection{Signal versus noise}

The primary aim of transcriptomics is to capture the expression patterns of the full set of transcripts in a cell or tissue. However, technical and biological variances cause fluctuations in the measurements, which is referred to as noise. Technical variation may result from the sequencing technology and from carrying out the experimental protocol; this type of noise is predictable and follows a Poisson distribution. Biological variation arises from the natural variability in different biological entities; this variation can be small for genetically identical samples or can be very large, such as in a heterogeneous sample such as blood. Biological variation may be due to biological processes that are not perfect such as noisy splicing, which increases the mRNA isoform diversity in human cells\cite{pmid21151575}. Other sources of noise may be from contamination, such as from genomic DNA.

The random sampling of lowly expressed versus highly expressed transcripts.

Expression profiles from sample of cells versus single cell.

\subsection{Coding versus non-coding transcripts}

Transcripts of Unknown Function (TUFs)
The role of non-coding RNAs
The role of repetitive elements in the genome

\subsection{Transcriptome profiling using CAGE}

Unbiased profiling of total transcripts
Comparison with different environmental conditions
Gene ontology enrichment
