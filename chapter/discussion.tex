\setlength{\parskip}{\baselineskip}%
\setlength{\parindent}{0pt}%

Small non-coding RNAs have been implicated in many biological processes, such as messenger RNA regulation or transposon silencing. We identified a role of small non-coding RNAs in the DNA damage repair mechanism. The inactivation of dicer and drosha, which are required for the biogenesis of small RNAs, leads to a loss of DNA damage repair. We sequenced the small RNAs from cells that had induced DNA damage and found small RNAs arising from the vicinity of the DNA double-strand break. Importantly, synthetic small RNAs mimicking these small RNAs could drive the DNA damage response \cite{francia2012site}.

Parkinson’s disease (PD) is a slowly progressive disease in which dopamine neurons in the substantia nigra degenerate undetected for years before clinical symptoms develop. The lack of clinical symptoms highlights the necessity of a laboratory test, such as an assay for biomarkers, which can correlate subjects with PD risk. We have profiled the RNAs in the whole blood sample of PD patients and age-matched controls using high-throughput deepCAGE sequencing. By comparing the RNA profiles between PD patients and controls, we aim to discover novel biomarkers that are present in whole blood, which may be further developed into a non-invasive clinical test for PD.

\subsection{Experimental differences}

poly-A versus random primers
cytoplasmic versus nuclear enrichment
CAGE versus RNA-seq

\subsection{Signal versus noise}

biological variance and technical variance
expression profiles from sample of cells versus single cell
highly expressed versus lowly expressed transcripts

\subsection{Coding versus non-coding transcripts}

Transcripts of Unknown Function (TUFs)
The role of non-coding RNAs
The role of repetitive elements in the genome

\subsection{Transcriptome profiling using CAGE}

Unbiased profiling of total transcripts
Comparison with different environmental conditions
Gene ontology enrichment
