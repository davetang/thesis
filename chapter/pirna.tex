Mutations in the gene methyl CpG binding protein 2 (MECP2) are the cause of most cases of Rett syndrome (RTT), a neurodevelopmental disorder that affects girls almost exclusively. Given that the MECP2 protein contains a methyl-CpG binding domain (MBD) and can recruit histone deacetylases, they are largely considered as transcriptional repressors. In rodents, the absence of MeCP2 leads to an increase in long interspersed nuclear elements-1 (L1) neuronal transcription and retrotransposition \citep{pmid21085180}. Furthermore, by analysing neuronal progenitor cells (derived from fibroblasts using induced pluripotent stem cell technology) and post mortem human tissues from patients with RTT, it was revealed elevated L1 activity compared \citep{pmid21085180}. MeCP2 chromatin immuno-precipitation sequencing in mice whole brain also suggested that MeCP2 may play a role in silencing retrotransposons \citep{pmid20188665}. A class of small RNAs called piwi-interacting RNAs (piRNAs), can be derived from L1 elements, and are involved in silencing transposons, by forming the piRNA-induced silencing complex (piRISC). In this work, we\footnote{All the bioinformatic analysis was performed by Dave Tang; details of the analysis can be found at \url{https://github.com/davetang/22976001}.} analysed a short RNA library prepared from cerebellum of mice with a Mecp2 knock-out \citep{pmid20921386}. Our analyses revealed that hundreds of piRNAs are expressed in the cerebellum, including previously characterised brain-specific piRNAs. Furthermore, we found that in the Mecp2 knock-out mice, a large number of piRNAs were over-expressed, suggesting that piRNA expression levels may also be affected by the absence of Mecp2. While piRNAs are thought to function mainly as silencers for transposons in germ cells, this study suggests that a subset of piRNAs may have other roles in gene regulation.
