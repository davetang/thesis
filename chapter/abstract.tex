The mammalian genome encodes both coding and non-coding transcripts that work in concert to orchestrate a wide range of cellular functions. By profiling the complete set of transcripts expressed, i.e., the transcriptome, the dynamics between transcripts can be inferred. One of the first technologies that allowed the simultaneous profiling of thousands of transcripts was microarrays and with the development of whole-genome arrays, the whole repertoire of known transcripts could be assayed. However, with the advent of high-throughput sequencers, RNA sequencing (RNA-Seq) became an attractive alternative to microarrays. RNA-Seq is an application of high-throughput sequencing that is used to study RNA expression levels in a discovery-based manner, since it does not rely on any \textit{a priori} knowledge of transcripts. RNA-Seq gives digital gene expression (DGE) profiles, which provides a large dynamic range of expression values and is much more quantitive than microarrays. Furthermore, RNA-Seq does not suffer from cross-hybridisation issues when profiling transcripts from repetitive regions of the genome.

RNA-Seq is relatively new compared to microarray technology, which has been around since 1995, but is gaining more widespread use due to the rapid drop in sequencing costs. As such RNA-Seq protocols are continually being developed and improved upon to avoid any biases. Methods used for storing, analysing, and visualising RNA-Seq data are also being heavily researched upon. In the near future, RNA-Seq may be used as routinely as microarrays for performing transcriptome profiling.

Non-coding RNAs, which include small and long non-coding RNAs, contribute to a significant portion of the transcriptome in mammalian genomes. However, the functional significance of this wide-spread occurrence of ncRNAs for organismal development and differentiation is unclear. While coding transcripts serve mainly as templates for protein synthesis, the functions of some well known non-coding transcripts are diverse. Furthermore, the major portion of many mammalian genomes are occupied by DNA sequences that are repetitive.

Technologies for the interrogation of nucleic acids have made it possible to investigate different biological questions.
