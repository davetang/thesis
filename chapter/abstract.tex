\begin{itemize}

   \item The mammalian genome encodes both coding and non-coding transcripts that work in concert to orchestrate a wide range of cellular functions.
   \item Genome-wide technologies that profile the complete set of transcripts, i.e., the transcriptome, have allowed researchers to better understand the dynamics between transcripts.
   \item The transition from microarraies to RNA sequencing (RNA-Seq) have unveiled a much more convoluted story.
   \item While coding transcripts serve mainly as templates for protein synthesis, the functions of some well known non-coding transcripts are diverse.
   \item The major portion of many mammalian genomes are occupied by DNA sequences that are repetitive.
   \item Non-coding RNAs, which include small and long non-coding RNAs contribute to a significant portion of the transcriptome in mammalian genomes.
   \item However, the functional significance of this wide-spread occurrence of ncRNAs for organismal development and differentiation is unclear.
   \item Technologies for the interrogation of nucleic acids have made it possible to investigate different biological questions.

\end{itemize}
