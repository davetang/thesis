The development of DNA sequencing has allowed us to fully determine the genetic make-up of all organisms. Technological innovations in DNA sequencing from the last two decades has massively scaled up the process of DNA sequencing in a time- and cost-effective manner and these advancements made it possible to fully sequence the genomes of various organisms, including humans. With the completion of various genome projects, the focus has now shifted towards identifying functional elements in genomes. In particular, the field of transcriptomics aims to identify, annotate, and analyse the transcribed products of the genome. Traditionally, transcriptome profiling has relied on microarray technologies but with the advent of high-throughput transcriptome sequencing, transcripts can be profiled without \textit{a priori} knowledge of their sequence and their expression levels can be digitally measured. This has transformed the field of transcriptomics, however the applications of high-throughput sequencing to transcriptomics is relatively immature as developments have occurred in the last 5-6 years. This thesis focuses on the interpretation of high-throughput transcriptome sequencing data through the use of bioinformatic methods.

In the first study of this thesis, biases and technical artefacts in a transcriptome technology known as nano cap analysis gene expression (nanoCAGE) were investigated. We found that biases were introduced through the use of molecular barcodes and developed several strategies for coping with such biases. In the second study, we captured and analysed the transcriptional output of cells with induced DNA damage using small RNA sequencing. We observed a previously uncharacterised class of small RNAs that formed near the DNA break site that were necessary in establishing the DNA damage response. In the third study, we studied the expression of a class of small RNAs called Piwi-interacting RNA (piRNAs) in mice with a Mecp2 knock-out (KO). Our analysis indicated that piRNAs are over-expressed in the cerebellum of mice with a Mecp2 KO, which may be due to proposed role of Mecp2 in silencing transposons. In the fourth and last study, we focused on transcripts that initiated from repetitive elements (REs) and characterised their expression patterns across a wide panel of cell lines, tissues, and primary cells. We demonstrated that REs may drive the expression of long non-coding RNAs and enhancer RNAs, and their expression patterns are tissue specific.

High-throughput sequencing has transformed the field of transcriptomics by revealing a much more complex picture of transcription than previously anticipated. The interpretation of this complexity has been a highly controversial topic; on one end of the spectrum, is the theory that the majority of the genome is functional, and on the other end is the counter-argument that most of the genome produces non-functional products. An important point raised from this discussion was that simply observing a transcription event is not enough evidence for function. Transcriptional signal may come from the expression of various classes of RNA species, or from random transcription events and technical artefacts. It is important to keep these considerations in mind when analysing eukaryotic transcriptomic data sets.
