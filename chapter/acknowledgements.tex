\epigraph{``Twenty years from now you will be more disappointed by the things that you didn't do than by the ones you did do. So throw off the bowlines. Sail away from the safe harbor. Catch the trade winds in your sails. Explore. Dream. Discover."}{--- \textup{Mark Twain}}

The culmination of work presented in this thesis would not have been possible without the guidance and support of many friends and colleagues throughout the years. I had always wanted to commence a PhD but my first attempt at securing a scholarship failed as I was ``not competitive enough". Thus I began a journey to become competitive. I was taken in as a work experience trainee under the auspices of the Commonwealth Scientific and Industrial Research Organisation (CSIRO). I am forever grateful to Brian who accepted me into his group and to Wes, who took me under his wing and trained me in the dark arts (a.k.a Perl). The training I received at CSIRO helped me land a job in Sean's lab, where I started my next stint as a LIMS developer/maintainer. It was there I learned of the PhD opportunity in Japan; I am extremely thankful to Geoff who forwarded the opening to me and convinced me to apply. I still remember the predicament I faced then when I had to decide whether or not I would commit to the idea of potentially working in Japan. I had brushed off the idea several times but in the end I was convinced that it was a tremendous opportunity and I would have regretted it if I let it slip by. In the end, I decided to set my sails to explore the world.

To this day, I have absolutely no regrets for embarking on this journey and it has been the best experience of my life (thus far). I had to leave behind close friends of many years but I still remember their responses when I asked for their advice regarding the opportunity to work in Japan. Their answers were unanimous; clearly, they had had enough of me. I would like to sincerely thank my supervisor, Piero, for his excellent support, patience, and trust throughout the entire PhD. Honestly, I wouldn't have trusted myself as much as he trusted me. I would like to thank Alka, for being my cheerful desk buddy (and mother figure), close friend, and colleague. I have learned so much from you by working on projects together and by prodding you with questions. I would like to thank Charles, for his sage advice on molecular biology, bioinformatics/informatics, statistics, and perhaps everything science related! He fits my definition of a guru more than anyone else I have ever met. I would like to thank Michiel, for all his statistical advice; sometimes I wish I could have appreciated more the advice he gave me not because I wasn't thankful but because it was statistics. I would like to thank Alex for reading through my thesis several times and helped me improve it significantly. I would also like to thank Charles (again), Michael, Marina, Quan, Kawaji-san, Kelly, and Cornelis for reading my thesis draft and offering feedback.

Several people have made my stay in Japan and RIKEN much more enjoyable. I would like to thank Raymond for being the first and only person I knew in Japan when I first arrived. You helped me settle in and showed me around; I wish I could have repaid your kindness. I would like to thank Magherita for being my original cheerful desk buddy; you always livened up the atmosphere. I would like to thank Marina (again) for being a close friend and for continually inviting me to events (despite my extremely low turn up rate). I would like to thank Morana for being a close friend, and for being the other foreign PhD student at our center and also the only other person interested in playing basketball (until the arrival of new recruits). I would like to thank Yuki for being a close friend, colleague, and partner in crime. I would like to thank Charles (again), Timo, and Al for the Friday meetings and the conservations that ensured. I would also like to thank Noro-san, who I thank for being the unofficial party organiser and for coming up with my nickname in RIKEN, and to Filip for his friendship and continual encouragement.

I was also fortunate enough to be able to travel to Europe on several occasions as part of my PhD and made some new friends. I would like to thank Stefano, Paolo, Francesca, Lavinia, Andreas, and Cornelis (again) for making my stay in Trieste very enjoyable. I would like to thank the entire BrainTrain crew for their friendship, companionship, and hospitality; in particular, I would like thank Min and Tiberiu. Finally, I would like to acknowledge the EU-ITN BrainTrain program, which sponsored this work.
