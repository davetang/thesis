It is well established that the human genome is pervasively transcribed and this has been attributed to a ``leaky" transcriptional system. Our observation that small RNAs are formed near the vicinity of DNA damaged sites and are required for establishing the DNA damage response, implies that all RNAs are potentially useful, since any site in the genome is vulnerable to DNA damage. A class of RNA that is often neglected from study are those that form from the repetitive portion of the genome. It is estimated that at least half of the human genome is made up of repetitive elements (REs) and transcription initiation has been observed within these elements \citep{pmid19377475}. However, the general impression is that these elements are non-functional and are simply transcriptional noise. Motivated by the hypothesis that all RNAs could be potentially useful, we quantified and catalogued expression signal from REs using the FANTOM5 atlas\footnote{Details of the analysis can be found at \url{https://github.com/davetang/fantom5_repeat}.}.
