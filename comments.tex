\documentclass[12pt,a4paper,sans]{article}
\usepackage{dirtytalk}

\begin{document}

\section{Abstract}

\begin{itemize}
   \item{Abstract: high throughput sequencing is not unbiased. It is definitely more comprehensive than microarrays.}
\end{itemize}
\noindent

I have re-worded the original sentence:
\\\\
\say{Traditionally, transcriptome profiling has relied on microarray technologies but with the advent of high-throughput sequencing, the ability to profile transcripts accurately and in an unbiased manner has transformed the study of transcriptomes.}
\\\\
into
\\\\
\say{Traditionally, transcriptome profiling has relied on microarray technologies but with the advent of high-throughput sequencing, the ability to profile transcripts the ability to profile transcripts accurately and in an unbiased manner has transformed the study of transcriptomes.}

\begin{itemize}
   \item{Abstract: last sentence needs rephrasing}
\end{itemize}

\begin{itemize}
   \item{Nederlandse samenvatting: tissue --> weefsel}
\end{itemize}

\section{Chapter 1}

\begin{itemize}
   \item{The description of SAGE is a bit inaccurate as the sequences in the early SAGE method were indeed always 9-10 bp long but the tag is usually said to also contain the restriction enzyme site so the total length is 13-14 and therefore better able to discriminate transcripts than a 9-10 bp tag. Also, the tag is not at the ultimate 3’-end as the text suggests.}
\end{itemize}

\begin{itemize}
   \item{The CAGE figure could be clearer. Currently it is not clear that only capped RNAs are measured, and it is not clear the endonuclease cuts downstream of its recognition site.}
\end{itemize}

\begin{itemize}
   \item{Section 1.5.4: spliceosomal RNAs aid protein translation?}
\end{itemize}

\begin{itemize}
   \item{Section 1.8: the research goal is very broad. It would be good to set a few more specific goals related to the different chapters}
\end{itemize}

\section{Chapter 3}

\begin{itemize}
   \item{I would advise to include supplementary data files. For example: Suppl Figures 22 and 23 are probably key figures for Chapter 3 generated by Dave but not included in the thesis.}
\end{itemize}

\section{Chapter 5}

\begin{itemize}
   \item{I miss a discussion on the non-capped RNAs that may also be . I also miss whether you observe clear CAGE peaks / start sites for these lncRNAs (like for protein-coding RNAs) or multiple distinct start sites or an even more even coverage pattern.}
\end{itemize}

\begin{itemize}
   \item{I miss also an analysis what kind of sRNAs are derived from these different repeat elements. Given the previous chapter, I would at least assume that you checked for Piwi RNAs, but I assume as well that most of these sRNAs are not capped so I am still a bit unclear what this enrichment in sRNAs actually means.}
\end{itemize}

\section{Chapter 6}

\begin{itemize}
   \item{I would reference to the specific chapter instead of to the paper (Chapter 6 instead of Tang et al 2013).}
\end{itemize}

\begin{itemize}
   \item{Title 6.0.3: MecP2: strange use of capitals and lower case.}
\end{itemize}

\begin{itemize}
   \item{6.0.3: when you refer to the gene, use Italics.}
\end{itemize}

\end{document}
